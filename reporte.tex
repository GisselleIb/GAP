\documentclass[12pt]{article}
\usepackage[paper=a4paper,margin=1in]{geometry}
\usepackage[spanish]{babel}
\usepackage[utf8]{inputenc}
\usepackage{enumerate}
\usepackage[ruled,vlined]{algorithm2e}
\usepackage{mathtools}
\usepackage{listings}
\usepackage{tikz}
\usepackage{float}
\usepackage{biblatex}
\addbibresource{ref.bib}


\author{Ibarra Moreno Gisselle }
\title{Problema de Asignación Generalizado (GAP)\\Heurísticas}
\begin{document}
	\maketitle
	
	\section{Introducción}
	\subsection{Generalized Assignment Problem}
	El problema del que se hablará en este reporte es el de \textbf{GAP}. Consiste en lo
	siguiente:
	
	Dadas $n$ tareas y $m$ trabajadores o agentes, se quieren asignar todas las tareas
	a los trabajadores. Cada trabajador tiene cierta energía o $capacidad$ que se va
	gastando conforme se le asignan tareas, cada tarea gasta un valor de $capacidad$
	dependiendo del trabajador al que se asigne, esto se puede ver como cuanta "dificultad" le puede costar al trabajador realizarla.
	Además cada tarea tiene un $costo$ que depende del trabajador que la vaya realizar.
	
	La $capacidad$ total del trabajador, la capacidad que gasta una tarea y el $costo$ 
	se pueden ver como las siguientes funciones:
	\[capacidadTotal=b(w)\]
	\[capacidad=a(t,w)\] 
	\[costo=\sum_{t=1}^{n}\sum_{w=1}^{m}c(t,w)\times x(t,w)\]
	Donde $w$ es un trabajador, $t$ es una tarea y $x(t,w)$ es la función que 
	representa si la tarea $t$ está asignada al trabajador $w$, es decir:
	\[x(t,w)=
	\begin{cases}
		\text{1}&\quad\text{si $w$ realiza la tarea $t$}\\
		\text{0}&\quad\text{e.o.c}
	\end{cases}\]
	
	Lo que busca optimizar el problema GAP es la función de costo.
	Además, una solución es factible si y sólo si se cumplen las siguientes 
	condiciones:
	\begin{enumerate}
		\item Toda tarea debe estar asignada a algún trabajador.
		\[\forall t \exists w (x(t,w)=1)\]
		\item Ninguna tarea debe estar asignada al mismo tiempo a dos o más 
		trabajadores.
		\[\forall t \neg \exists w_1,w_2 (x(t,w1)=1 \;\wedge \; x(t,w_2)=1)\]
		\item Un trabajador no debe superar su $capacidadTotal$.
		\[\forall w \left(\sum_{t=1}^{n}a(t,w)\times x(t,w) \; \leq b(j)\right)\]
	\end{enumerate}

	Para resolver este problema se decidió utilizar la heurística \textit{Colonia de 
	Abejas, (Bee Colony Optimization)}.

  	\subsection{Bee Colony Optimization}
  	
  	Esta heurística está basada en la inteligencia de enjambre que tienen las abejas
  	para encontrar buenas fuentes de comida.
  	La inteligencia enjambre se refiere al comportamiento de organismos que siguen
  	reglas simples para interactuar con su medio y otros organismos, estas 
  	interacciones derivan en comportamientos más complejos que pueden resolver 
  	problemas complicados.
  	
  	Como se mencionó esta heurística está basada en el comportamiento de un enjambre 
  	de abejas en la naturaleza. La búsqueda y recolección del néctar es un proceso
  	complejo que necesita demasiada organización en las abejas. Se tienen abejas 
  	exploradoras que salen en busca de flores de las cuales recolectar néctar, cuando
  	alguna de estas abejas encuentra una fuente de alimento, regresa al panal para
  	comunicarselo a las demás. La abeja procede a hacer un baile que indica
  	la posición y distancia de la comida, al igual que la calidad. De esta forma las 
  	otras abejas aprenden la localización aproximada del alimento y van en busca de 
  	él. Así las abejas logran	la organización para que varios grupos de
  	abejas recolecten el néctar de diferentes fuentes de alimento y se concentren en 
  	las mejores.
	
  	La heurística toma este comportamiento de forma simplificada para resolver
  	problemas de combinatoria. El proceso de búsqueda se compone de iteraciones, en 
  	cada una de ellas se construyen soluciones de las cuales guardamos la mejor.
  	Cada solución construida en la iteración $i$ es independiente de las soluciones
  	construidas en la iteración $i-1$ o $i+1$.
  	Todas las abejas se encuentran en el panal al inicio
  	del proceso de búsqueda y se comunican directamente entre ellas. Durante una 
  	iteración, cada abeja puede hacer una serie de movimientos locales, cada uno de 
  	estos movimientos construye parte de una solución, este proceso de construcción
  	se lleva a cabo con dos fases que definen el comportamiento de las abejas:
  	\textit{forward pass} y \textit{backward pass}.
  	
  	\begin{enumerate}
  		\item \textit{Forward pass:} En esta fase se mandan a las abejas en el último
  		punto en que se quedó su solución parcial, a partir de ahí generan un 
  		movimiento local que agrega una parte más a la solución.
  		\item \textit{Backward pass:} En esta fase las abejas regresan al panal y 
  		comparan sus soluciones entre todas, cada abeja tiene una lealtad hacia su 
  		solución que depende de la calidad de esta. Cada abeja tiene tres opciones:
  		\begin{itemize}
  			\item Reclutar abejas seguidoras hacia su solución para que ellas expandan 
  			soluciones parciales a partir de ella.
  			\item Volver a su solución sin reclutar ninguna abeja.
  			\item Abandonar su solución y seguir a alguna de las abejas que reclutan.
  		\end{itemize}
  	\end{enumerate}
  	
  	Estas fases se repiten hasta obtener al menos una solución factible. La 
  	ejecución de estas dos fases se puede ver como estados, cada uno de estos
  	estados involucra una variable a optimizar, esto se puede ver como:
  	\[ST=\{st_1,st_2,...,st_m\}\]
  		
  	A continuación se puede ver el pseudocódigo para la heurística:
  		
  	\begin{algorithm}[H]
  		\SetAlgoLined
  		\KwOut{$s$ solución encontrada}
  		\KwIn{Una colonia B de abejas, $I$ iteraciones,conjunto de estados $ST$, 
  			$s$ una solución inicial factible}
  		\For{i=1; i= I }{
  		   \For{j=1; j=m}{
  		   	$ForwardPass(B,st_j)$\\
  	   	    $BackwardPass(B)$}
       	   $s_i=getBestSolution(B)$\\
              \If{$s_i < s$}{
         	      $s=s_i$}}
  		\Return{$s$}
  		\caption{Bee Colony Optimization}
  	\end{algorithm}
  
  	\begin{algorithm}[H]
  		\SetAlgoLined
  		\KwIn{Una colonia B de abejas,un estado $st$ }
  		\ForEach{bee in B}{
  			$bee.localSearch(st)$
  		}
  		\caption{ForwardPass}
  	\end{algorithm}
  
  	\begin{algorithm}[H]
  		\SetAlgoLined
  		\KwIn{Una colonia B de abejas}
  		$B.compareSolutions()$\\
  		\ForEach{bee in B}{
  			$bee.changeStatus()$ 
  		}
  		\caption{BackwardPass}
  	\end{algorithm}
  		
  	\paragraph{}
  	Donde $localSearch()$ es una función que expande la solución parcial de la abeja
  	con movimientos locales, $compareSolution()$ hace la comparación de todas las 
  	soluciones de las abejas generando la lealtad de cada abeja hacia sus soluciones
  	y $changeStatus()$ es la decisión que toma la abeja respecto a su solución 
  	tomando en cuenta su lealtad.
  	
  	
	\section{Desarrollo}
	
	\subsection{Tecnologías Utilizadas:}
	
	El lenguaje de programación utilizado para la implementación del algoritmo, es Nim.
	Es un lenguaje completamente compilado, multiparadigma con tipado estático. 
	
	El compilador genera código optimizado en C y también puede generar código en C++ y
	JavaScript.
	El compilador además tiene un recolector de basura y el programador no tiene porque
	manejar la memoria de forma manual y directa.
	
	La sintáxis de Nim es muy parecida a la de Python lo que la hace muy sencilla de
	comprender y leer. Además traducir pseudocódigo a código en Nim, es igualmente
	sencillo y rápido.
	
	\subsection{Implementación de la Heurística:}
	
	Se implementó la heurística de la Colonia de Abejas adapatada para resolver el
	problema \textbf{GAP}. La heurística fue implementada utilizando el paradigma de
	Orientación a Objetos.
	
	En esta implementación las abejas construyen las soluciones desde 0 hasta llegar
	a una solución factible en una iteración.La forma en que construyen las soluciones
	las abejas es agregando una tarea a un trabajador de forma aleatoria hasta que no
	queden tareas sin asignar.
	
	Una parte primordial del problema es la capacidad que gasta cada trabajador con una
	tarea y el costo ya que es lo que queremos optimizar, estos valores se necesitarán
	constantemente a lo largo de toda la heurística.
	Las funciones $a(t,w), c(t,w)$ para la capacidad y el costo respectivamente se 
	pueden traducir como una sola matrix de $n \times m$ que almacenará una tupla con
	la capacidad y el costo de la tarea $i$ con el trabajador $j$, esto nos facilitará
	la obtención de cualquiera de estos valores cada que se necesiten en tiempo 
	constante.
	
	
	\paragraph{Trabajadores}
	
	Los trabajadores son un factor muy importante, ya que ellos tienen
	características importantes como la $capacidadTotal$ que pueden aguantar y 
	se necesita saber las tareas que tiene asignado cada trabajador, 
	mientras que las tareas no tienen realmente características y los valores 
	importantes relacionadas a ellas están almacenadas en nuestra matriz, por lo que 
	podemos	representar cada una de estas tareas con números enteros positivos.
	Por eso mismo se diseñó un objeto $Worker$ para representar a los trabajadores con
	los siguientes atributos y funciones para modelar su comportamiento:
	\begin{enumerate}
		\item \textbf{Atributos:}
		\begin{itemize}
			\item $totalCapacity$: La capacidad total que puede soportar el trabajador
			\item $capacity$: La capacidad que lleva gastada el trabajador
			\item $tasks$: Una lista de enteros que almacenará las tareas asignadas al 
			trabajador.
		\end{itemize}
		\item \textbf{Funciones:}
		\begin{itemize}
			\item $addTask$: Agrega una tarea a la lista $tasks$ y suma a $capacity$
			la cantidad de capacidad que gasta la nueva tarea asignada.
			\item $excessCapacity$: Indica si la capacidad gastada ha sobrepasado 
			a la capacidad total del trabajador $capacity > totalCapacity$
		\end{itemize}
	\end{enumerate}

	Con la clase para modelar a los trabajadores y una forma sencilla de 
	representar a las tareas, se modela la representación de las soluciones.
	
	\paragraph{Soluciones} Las soluciones son objetos que necesitan llevar de alguna
	forma registro de todos los trabajadores con las tareas que llevan asignadas hasta
	el momento, también necesita saber la cantidad de tareas que están sin asignar y
	lo más importante, necesitan llevar registro del costo.
	Sus atributos y funciones más importantes son:
	
	\begin{enumerate}
		\item \textbf{Atributos:}
		\begin{itemize}
			\item $workers$: Un diccionario que almacena todos los trabajadores y como 
			llave utiliza el identificador del trabajador y como valor un objeto 
			$Worker$.
			\item $tasksLeft$: Una lista de enteros que representan las tareas que 
			aún no han sido asignadas a ningún trabajador.
			\item $cost$: El valor de la suma de los costos de cada tarea que ha sido 
			asignada a algún trabajador.
		\end{itemize}
		\item \textbf{Funciones:}
		\begin{itemize}
			\item $addTaskToWorker$: Escoge un trabajador de forma aleatoria y le 
			asigna la siguiente tarea, además suma el costo de que ese trabajador 
			realice esa tarea al costo de la solución, pero si la capacidad total del
			trabajador ya fue excedida entonces suma el costo multiplicado por mil, de
			esta forma penalizamos cualquier solución parcial que ya no es factible.
		\end{itemize}
	\end{enumerate}
	
	Teniendo nuestra clase para las soluciones podemos empezar a ver el diseño que 
	corresponde a la Colonia de Abejas.
	
	\paragraph{Abejas}
	Como se dijo anteriormente, las abejas construyen las soluciones desde 0 agregando
	una tarea cada vez, además, con el $backwardpass$ las abejas pueden cambiar el 
	su estatus y su solución por alguna mejor en algún momento de la iteración.
	Podemos ver los atributos y funciones principales de la abeja a continuación:
	
	\begin{enumerate}
		\item \textbf{Atributos:}
		\begin{itemize}
			\item solution: La solución actual de la abeja.
			\item status: El estatus de la abeja, puede ser $follower$(seguidora), 
			$loner$ (solitaria) y $dancer$(bailarina). Este estatus determina 
			el comportamiento de la abeja.
		\end{itemize}
		\item \textbf{Funciones:}
		\begin{itemize}
			\item $loyalty$: Calcula la lealtad que tiene la abeja hacia su solución
			actual, esta se calcula como:
			\[\frac{S-S_{min}}{S_{max}-S_{min}}\]
			Donde $S$ es el costo de la solución de la abeja, $S_{min}$ es la solución
			con el costo más pequeño de toda la colonia y $S_{max}$ es la solución con
			el peor costo. Esta fórmula coloca el valor del costo en un rango entre 0 
			y 1 lo que facilita determinar que tan buena o mala es una solución.
			\item $changeStatus$: Determina el nuevo estatus de la abeja, esto lo 
			hace con ayuda de la lealtad que tenga la abeja hacia su solución. La forma
			en que lo determina es elitista, cualquier abeja con lealtad menor 0.05 
			se convertirá en una abeja bailarina que reclute, las abejas con lealtad
			mayor a 0.05 pero menor 0.2 serán las abejas solitarias que continuarán
			expandiendo su solución por su cuenta, las abejas con lealtad mayor o 
			igual a 0.2	serán abejas seguidoras.
		\end{itemize}
	\end{enumerate}

	\paragraph{Colonia de Abejas}
	La colonia de abejas tendrá como atributo una estructura con $k$ número de abejas, 
	el número de iteraciones que se ejecutará la heurística y la mejor solución 
	encontrada de todas las iteraciones ejecutadas. Además en esta clase es donde
	viene implementado el funcionamiento principale de la heurística como son las fases
	$forwardpass$ y $backwardpass$.
	Los atributos de la abeja y las funciones se muestran a continuación:
	
	\begin{enumerate}
		\item \textbf{Atributos:}
		\begin{itemize}
			\item $bees$:Un arreglo de tamaño $k$ con todas las abejas de la colonia.
			\item $iterations$: El total de iteraciones que se ejecutará la heurística.
			\item $bestSolution$: La mejor solución encontrada hasta la iteración $i$.
		\end{itemize}
		\item \textbf{Funciones:}
		\begin{itemize}
			\item $forwardpass$: Estando en el estado $j$, para cada abeja en $bees$ 
			manda llamar la función $addTaskToWorker$ de la solución asociada a la 
			abeja.
			\item $backwardpass$: Para cada abeja en $bees$ elige su estatus con 
			$changeStatus$, luego para cada abeja con estatus $follower$ elige de 
			forma aleatoria una de las abejas con estatus $dancer$.
			\item $beeColonyOpt$: Ejecuta la heurística tal como se vió en el 
			pseudocódigo en la introducción.  
		\end{itemize}
	\end{enumerate}

	http://graphonline.ru/en/?graph=mlwaCuHVLgyaUFrA
	
	\section{Resultados}
	
	Se probó la heurística con dos instancias de bases de datos distintas, se hicieron
	varias pruebas con 300 semillas para las dos instancias cambiando los valores en 
	el elitismo al cambiar el estatus de cada abeja.
	
	\begin{enumerate}
		\item Lealtad 1:
		\[loyalty < 0.05 = dancer,  0.05 < loyalty < 0.2 = loner,0.2<loyalty=follower\]
		\item Lealtad 2:
		\[loyalty < 0.02 = dancer,  0.02 < loyalty < 0.2 = loner,0.2<loyalty=follower\]
	\end{enumerate}
	
	La siguientes tablan los diferentes valores con los que se eligieron los estatus
	de las abejas:
	
	
	\begin{table}[H]
		\centering
		\caption{Lealtad 1}
		\begin{tabular}{|p{0.13\textwidth}|p{0.12\textwidth}|p{0.13\textwidth}|p{0.14\textwidth}|p{0.17\textwidth}|p{0.17\textwidth}|}
			\hline 
			Instancia de BD & Semilla & Número de Abejas & Iteraciones & Mejor solución  & Peor solución \\
			\hline 
			gap3.db & 284 & 60 & 20 & 3957.68204 & 229163.8117 \\
			\hline 
			gap3.db & 376 & 60 & 20 & 4082.93975 & 219452.8191 \\
			\hline 
			gap3.db & 8 & 60 & 20 & 4109.79893 & 217727.8319 \\
			\hline 
			gap4.db & 303 & 100 & 30 & 8651.52446 & 209384.7634 \\
			\hline 
			gap4.db & 79 & 100 & 30 & 8747.21108 & 248956.0246 \\
			\hline 
			gap4.db & 158 & 100 & 30 & 8812.41736 & 213476.8721 \\
			\hline
		\end{tabular}
		
	\end{table}

	
	\begin{table}[H]
		\centering
		\caption{Lealtad 2}
		\begin{tabular}{|p{0.13\textwidth}|p{0.12\textwidth}|p{0.13\textwidth}|p{0.14\textwidth}|p{0.17\textwidth}|p{0.17\textwidth}|}
			\hline 
			Instancia de BD & Semilla & Número de Abejas & Iteraciones & Mejor solución  & Peor solución \\
			\hline 
			gap3.db & 20 & 60 & 20 & 3935.82122 & 220149.7641 \\
			\hline 
			gap3.db & 228 & 60 & 20 & 4000.05095 & 219036.0994 \\
			\hline 
			gap3.db & 280 & 60 & 20 & 4030.89876 & 217541.7850 \\
			\hline 
			gap4.db & 209 & 100 & 30 & 8158.21757 & 205790.7171 \\
			\hline 
			gap4.db & 59 & 100 & 30 & 8258.10292 & 204717.3900 \\
			\hline 
			gap4.db & 221 & 100 & 30 & 8263.36881 & 202263.8813 \\
			\hline
		\end{tabular}
		
	\end{table}

	Se muestra a continuación la mejor configuración del problema para la base de datos
	gap3.db:
	
	\begin{lstlisting}
seed: 284
cost: 3957.682043498803

W1: T4,T9
W2: T14,T20
W3: T17,T30
W4: T39,T40
W5: T3,T16
W6: T36
W7: T47,T48
W8: T15,T33
W9: T13,T42
W10: T25,T41,T49,T50
W11: T5,T27
W12: T2,T9,T11,T45
W13: T32,T35
W14: T7,T21,T22,T43
W15: T6,T8,T10,T24
W16: T12,T28,T29
W17: T18,T34
W18: T5,T38
W19: T1,T23,T26,T37
W20: T31,T44,T46
	\end{lstlisting}

	La mejor configuración del problema para la base de datos gap4.db:

	\begin{lstlisting}
seed: 209
cost: 8158.217579479722

W1: T14,T15,T41,T49
W2: T12,T22,T40
W3: T32,T36,T48
W4: T6
W5: T87
W6: T9,T11.T35,T60,T80,T81,T94
W7: T8,T23
W8: T25,T29,T70,T72
W9: 
W10: T37
W11: T31,T82,T93,T95
W12: T24,T75,T78
W13: T88
W14: T7,T44
W15: T28
W16: T10,T50,T53
W17: T56,T68
W18: T1,T65,T90
W19: T34,T47,T62,T63
W20: T57,T61,T69,T71,T74
W21: T16,T54,T86
W22: T13,T64,T73,T99
W23: T17
W24: T52,T55
W25: T2
W26: T67,T83,T84
W27: T46
W28: T33,T85,T89,T97
W29: T3,T5,T18,T42,T51
W30: T20,T76
W31: T58
W32: T38,T66
W33: T21,T27,T92,T98
W34: T4,T45,T77
W35: T30
W36: 
W37: T43
W38: T19,T49
W39: T39,T79,T96,T100
W40: T26,T91
	\end{lstlisting}
	\section{Conclusión}
	Se lograron encontrar soluciones factibles, aunque los valores obtenidos no se 
	acercan a los valores óptimos.
	Debido a que las soluciones se construyen agregando tarea por tarea, no hay 
	posibilidad de mover alguna tarea hacia otro trabajador y que mejore todavía 
	más la solución. Como las abejas actuan de forma elitista, las mejores 
	soluciones se van construyendo por las abejas que no han sobrepasado la capacidad
	de algún trabajador, mientras que las que sobrepasan la capacidad de alguno
	es descartada automáticamente. Esto puede ser malo ya que no se le está dando
	suficiente oportunidad de empeorar a la solución. 
	
	\section{Referencias:}
	\begin{itemize}
		\item \textit{D. Teodorovic, P. Lucic, G. Markovic and M. D. Orco, "Bee Colony Optimization: Principles and Applications," 2006 8th Seminar on Neural Network Applications in Electrical Engineering, Belgrade, Serbia \& Montenegro, 2006, pp. 151-156, doi: 10.1109/NEUREL.2006.341200.}
		
		\item \textit{O Erhun Kundakcioglu and Saed Alizamir. Generalized assignment problem.,
			2009.}
		\item \textit{nim-lang.org}
	\end{itemize}
	
	
\end{document}